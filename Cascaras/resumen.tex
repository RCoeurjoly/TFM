\chapter*{Resumen}

\section*{\tituloPortadaVal}

Presentamos un depurador declarativo para C++, llamado DDC. Un depurador declarativo recibe como argumento de entrada una computación incorrecta, construye un árbol de depuración basado en la ejecución del programa y, después de preguntar a un oráculo (tipicamente el usuario), indica el fragmento de código causante del fallo. Presentamos las principales características del depurador, tales como tres estrategias de navegación, el uso de casos de prueba como oráculo, capacidad de depurar programas que no terminan y una transformación de árbol. 

\section*{Palabras clave}
   
\noindent depuración declarativa, C/C++, depuración inversa, verificación formal, Coq

   


