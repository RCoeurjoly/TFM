\chapter{Introduction}
\label{cap:introduccion}

\chapterquote{Program testing can be used to show the presence of bugs, but never to show their absence!}{Edsger Dijkstra}

\section{Motivation}

Debugging is the most expensive part of developing software.
It is estimated that between 80 and 90 percent of development effort is spent on debugging tasks.  (reference?)

Furthermore, no paradigm (imperative, functional, logical, etc) or language can claim that it removes the need to debug.

Therefore, it is important to develop tools and workflows to alleviate this issue.

Declarative debugging (also known as algorithmic debugging) takes a semiautomatic approach.

To the best of our knowledge, there is no declarative debugger for C++.

There are declarative debuggers for:
 \begin{itemize}
 \item Maude
\item  Java
\item Erlang
\end{itemize}

Developing a declarative debugger for C++ involves figuring out a way to treat pointers in a way useful for the user.

\section{Goals}
The goal of this thesis is to develop a declarative debugger for the C++ language.

This debugger, called DDC, should have the following features:
\begin{itemize}
  \item The debugger has to scale to real programs.
  \item Integrated in the existing debugging workflow of the developer.
  \item The user has to be able to debug a program with no or few changes to its compilation (at most setting some compilation flags).
\end{itemize}

scalability issue:

Provide a way for the user to choose what are the suspect functions or methods.
This would reduce the number of nodes in the execution tree (ET), therefore reducing the time to build it and its memory footprint.
Provide a way for the user to choose a point in the code that, if reached, triggers the end of the building of the ET and begins the asking questions to the user.

Use test cases to reduce the number of nodes in the tree, therefore reducing the number of questions needed to find the buggy node.

Usability:

Integrate the declarative debugger into the debugger most used by C++ developers.

This would provide several benefits:

The user would set the breakpoints (both the suspect function or methods and the final point) in a way that is identical to her usual debugging workflow.

Common features when setting breakpoints like auto-completion

The user would not have to switch tools between normal debugging and declarative debugging.
\section{Main contributions}

The main contribution of this thesis is the development of a declarative debugger for C++, called DDC.

The most notable characteristics of DDC are the following:
\begin{itemize}
\item Support for several programming languages, by means of using GDB
\item Non terminating programs can be debugged
\item Test cases can be used to reduce the tree size
\item 3 asking strategies developed
\item 1 tree transformation developed
\item easily extensible (more strategies, more tree transformations)
\end{itemize}