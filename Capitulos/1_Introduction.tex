\chapter{Introduction}
\label{cap:introduccion}

\chapterquote{}{}


In this section, we present the motivation of the project, its goals, contributions, and the structure of the rest of the document.\todo{Falta la cita al principio, poner o quitar.}

\section{Motivation}
C++ is a programming language that dates back to 1979 \cite{cppHistory}, and one of its goals is to maintain backward compatibility with C \cite{cCompat}. As such, C++ has inherited the defects of C, such as lack of memory safety \cite{practicalmemorysafety}.
Despite these shortcomings, C++ is still one of the most used programming languages \cite{popularityPL}, including being used in such important areas such as:
\begin{itemize}
    \item Compilers: GCC \cite{gcc}, LLVM \cite{llvm}.
    \item Databases: MySQL \cite{mysql}, MongoDB \cite{mongodb}.
    \item Theorem provers: Z3 \cite{z3}, Lean \cite{lean}.
    \item Debuggers: GDB \cite{gdb}, rr \cite{rr}.
    \item Machine learning frameworks: TensorFlow \cite{tensorflow}, PyTorch \cite{pytorch}.
    \item Digital currency and smart contracts technology: Bitcoin \cite{bitcoin}, Solidity \cite{solidity}.
\end{itemize}

Furthermore, C++ is actively developed, with the last release being C++20. This, together with the upward trending usage statistics \cite{popularityPL}, tends to indicate that the need to develop, test, and debug C++ programs is going to continue into the future.

This need to debug C++ programs is what lead to this Master's thesis.
Debugging is one of the most expensive part of developing software.
It is estimated that between 35 and 50 percent of development effort is spent on validation and debugging tasks \cite{debuggingMindset}.

There are many debugging approaches, ranging from the fully manual to the almost fully automatic, like delta debugging or program slicing \cite{WhyProgramsFail}. In this Master's thesis we focus on declarative debugging \cite{shapiro1982algorithmic}, also called algorithmic debugging, which is a semi automatic approach in which the debugger builds a debugging tree of the program and guides the user through it by asking questions about the correctness of sub-computations until a buggy function is found. We expand the definition of declarative debugging in Chapter \ref{cap:preliminares}.

Several declarative debuggers have been developed, including for object-oriented languages such as Java \cite{DDJ}, but to the best of our knowledge, there is no declarative debugger for C++. We review the State of the Art of declarative debugging and its implementations in Chapter \ref{cap:estadoDeLaCuestion}.

\section{Goals}
The goal of this Master's thesis is to develop a declarative debugger for the C++ language.
This debugger should have the following features:
\begin{enumerate}
  \item Integrated in workflow: it has to be integrated in the existing debugging workflow of the developer. \label{goal1}
  \item No changes to program: the user has to be able to debug a program with no or few changes the program and its compilation (at most setting some compilation flags). \label{goal2}
  \item Tree transformations: it has to perform tree transformations to reduce the size of the debugging tree. \label{goal3}
  \item Test cases as oracles: it has to use test cases to reduce the number of questions posed to the user. \label{goal4}
\end{enumerate}

\section{Main contributions}

The main contribution of this Master's thesis is the development of a declarative debugger for C++, called DDC.
%
The most notable characteristics of DDC are the following:
\begin{enumerate}
\item Support for several programming languages, by means of using GDB.
\item Non terminating programs can be debugged.
\item Test cases can be used as oracles to reduce the tree size.
\item Three navigation strategies have been developed.
\item One tree transformation has been developed.
\item It can be easily extended with more strategies and more tree transformations.
\end{enumerate}

The source code of the project is available at \url{https://github.com/RCoeurjoly/DDC} 
and its license is AGPL-3.0 License.

\section{Structure of the document}
The rest of the thesis is organized as follows:
\begin{itemize}
    \item In Chapter \ref{cap:preliminares} we introduce declarative debugging and the different tools needed to implement DDC.
    \item  In Chapter \ref{cap:estadoDeLaCuestion} we review the state of the art in declarative debugging. 
    \item Chapter \ref{cap:definiciones} contains the definitions and proofs that form the theoretical basis of DDC. 
    \item Chapter \ref{cap:toolDescription} presents the most important usage scenarios of the tool, describes how the debugger was implemented, lists its commands, and compares its features to other declarative debuggers. 
    \item Chapter \ref{cap:conclusions} discusses the achievements and the future lines of work that can be followed.
\end{itemize}
