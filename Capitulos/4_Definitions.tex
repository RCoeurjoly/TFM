\chapter{Definitions}
\label{cap:definiciones}

\todo{No empieces la sección con una subsección, pon algo de texto
explicando de qué va a ir.}

\section{Node}\todo{Usar entorno definition o algo así. Cuando una definición salga de un artículo pon la cita en el nombre de la definición (esto ya para las siguientes)}
A node is a function/method execution.
It contains:
\begin{itemize}
\item The name of the function/method executed\todo{Punto despúes de cada item}
\item The input arguments when it was called (if any)
\item The object state when it was called (if it is a method call)
\item The global variables state when it was called (if any)
\item The output arguments when it returned (if there where passed as reference or pointer)
\item The object state when it returned (if it is a method call)
\item The global variables state when it returned (if any)
\end{itemize}
\section{Weighted Marked Execution Tree}
We here expand the definition of MET in "An Optimal Strategy for Algorithmic Debugging" to WMET.\todo{Las comillas en latex son ``asi''}\todo{falta referencia}

A weighted marked execution tree (WMET) is a tree T = (N,E,W,M) where N are the nodes,E⊆N×N\todo{Usar entorno matemático y los comandos correspondientes en todas las fórmulas} are  the edges, M:N→V is a total function that assigns to all the nodes in N a value in the domain D={Wrong,Undefined} and W is a total function that assigns to all the nodes in N a value which is the weight  of  the  sub-tree  rooted  at  node n in N,wn,  is  defined recursively as its number of descendants including itself (i.e.,1 +∑{wn'|n→n′)∈E}).
\section{Buggy node}
A buggy node is a node that is marked as incorrect by the user.

At first we considered inferring that the root node of the tree, upon completing of the building of the tree, is wrong, since the user started a debugging session.

Upon closer analysis we decided against doing so, to provide more flexibility to the user.
\section{Detected errors}
DDC can help the user detect the following errors:
\begin{itemize}
\item A function/method was called with the wrong arguments
\item A function/method returns a wrong value
\item A global variable is visible from the function/method scope when it should not 
\item A global variable is not visible from the function/method scope when it should 
\item A method modifies its object when it should not 
\item A method does not modify its object when it should do so.
\item A function/method modifies an argument passed by reference or pointer when it should not 
\item A function/method does not modify an argument passed by reference or pointer when it should do so
\item An argument passed by reference or pointer has the wrong value when returning.
\item A function/method does not terminate when it should do so
\item A function/method calls a sub-computation when it should not.
\item A function/method does not call a sub-computation when it should do so.
\end{itemize}
\section{Completion of the debugging session}
The debugging session is finished when:
The WMET is empty, therefore no buggy node has been found
The WMET consists of one node marked wrong, therefore the buggy node has been found.
\section{Correctness}