\chapter{Verification and benchmarks}
\label{cap:verificationAndTesting}

In this chapter, we present the formal verification done in Coq and time benchmarks of the execution of DDC.\todo{Contar algo de Coq en los preliminares}

\section{Verification}
To help with the verification effort, we have chosen the Coq proof assistant \cite{coq}. The proofs can be found in the following file: \url{https://github.com/RCoeurjoly/DDC/blob/database/proofs.v}.

We now present the types, functions, and lemmas defined in Coq.

\subsection{Types}
We have defined two inductive types:
\begin{itemize}
    \item Correctness (see Figure \ref{fig:correctnessType}): a simple inductive type, with the possible values presented in Chapter \ref{cap:preliminaries}.
    \item Node: an inductive recursive type, whose constructor, \verb|mkNode|, takes as arguments:
    \begin{itemize}
        \item a string to represent the content (\(f\), \(I\) and \(O\) in Chapter \ref{cap:definitions}) of the node,
        \item a \verb|Correctness| value and
        \item a list of \verb|Node|s.
    \end{itemize}
     
\end{itemize}
\begin{figure}
    \centering
    \caption{Correctness type}
    \label{fig:correctnessType}
\begin{minted}{coq}
Inductive Correctness : Type :=
| yes : Correctness
| no : Correctness
| trusted : Correctness
| idk : Correctness.
\end{minted}
\end{figure}

\begin{listing}[!ht]
\begin{minted}{coq}
Inductive Node : Type :=
  mkNode
    {
    content : string
    ;correctness : Correctness
    ; children : list Node
    }.
\end{minted}
\caption{Node inductive type}
\label{listing:2}
\end{listing}

\subsection{Functions}
Apart from the generic debugging algorithm, we have designed some auxiliary functions needed to reason about debugging trees:

\begin{itemize}
\item \verb|or_list|: recursive function to calculate the disjunction of a list of propositions.
\item \verb|and_list|: recursive function to calculate the conjunction of a list of propositions.
\item \verb|is_node_in_tree|: recursive function to find if the content of a node is in a tree.
\item \verb|weight|: recursive function to calculate the weight of a tree, inspired by the definition in \cite{optimal_strategy}.
\item \verb|are_all_idk|: recursive function to check that all nodes in tree have a correctness value of \verb|I don't know|.
\item \verb|is_debugging_tree|: check if a tree is a debugging tree, defined as the root being incorrect and all children undefined (\verb|I don't know|) recursively.
\item \verb|get_debugging_tree_from_tree|: returns the same tree as the input, only changing the correctness of the root node to incorrect.
\item \verb|generic_debugging_algorithm|: a dependently-typed program, which takes as input a debugging tree and finds the buggy node. We provide a measure, the weight of the input tree, which we must prove it decreases with each recursive call to prove termination of the algorithm.
To simplify the proofs, the navigation strategy used is Top-down and the user always answers that the node is incorrect, therefore choosing the first child as the next focus of the algorithm.
\end{itemize}

\begin{listing}[!ht]
\begin{minted}{coq}

Fixpoint is_node_in_tree (c : string) (m : Node) : Prop :=
  or (c = content m)
     (or_list
        (map
            (fun child => is_node_in_tree c child)
                (children m))).

\end{minted}
\caption{is\_node\_in\_tree function}
\label{listing:3}
\end{listing}

\begin{listing}[!ht]
\begin{minted}{coq}

Fixpoint and_list (l : list Prop) : Prop :=
  match l with
    nil => True
  | hd::tl => and hd (and_list tl)
  end.

\end{minted}
\caption{and\_list function}
\label{listing:4}
\end{listing}

\begin{listing}[!ht]
\begin{minted}{coq}
Fixpoint weight (node : Node) : nat :=
  match children node with
    nil => 1
  | children => S (list_sum (map (fun child => weight child) (children)))
  end.
\end{minted}
\caption{weight function}
\label{listing:4}
\end{listing}

\begin{listing}[!ht]
\begin{minted}{coq}
Fixpoint are_all_idk (node : Node) : Prop :=
  and
    (node.(correctness) = idk)
    (and_list
        (map
            (fun child => are_all_idk child)
                (children node))).
\end{minted}
\caption{are\_all\_idk function}
\label{listing:4}
\end{listing}


\begin{listing}[!ht]
\begin{minted}{coq}
Definition is_debugging_tree (node : Node) : Prop :=
  and
    (node.(correctness) = no)
        (and_list
            (map
                (fun child => are_all_idk child)
                    (children node))).
\end{minted}
\caption{is\_debugging\_tree function}
\label{listing:4}
\end{listing}


\begin{listing}[!ht]
\begin{minted}{coq}
Definition get_debugging_tree_from_tree (n : Node) : Node :=
  mkNode (content n) no (children n).
\end{minted}
\caption{get\_debugging\_tree\_from\_tree function}
\label{listing:4}
\end{listing}


\begin{listing}[!ht]
\begin{minted}{coq}
Program Fixpoint generic_debugging_algorithm
    (n : {n: Node | is_debugging_tree n})
    {measure (weight n)}:
    {m: Node | is_debugging_tree m /\ children m = nil} :=
  match children n with
    nil => n
  | head::tail => generic_debugging_algorithm
    (get_debugging_tree_from_tree head)
  end.
\end{minted}
\caption{generic\_debugging\_algorithm function}
\label{listing:4}
\end{listing}

\subsection{Lemmas}
The main goal of our verification effort is to confirm that the general debugging algorithm defined in Figure \ref{} type checks. To do this, we have to discharge four proof obligations:
\begin{enumerate}
    \item The algorithm returns a debugging tree without children nodes when it receives as input a debugging tree without children nodes.
    
    This is proved by simplification, that is, execution, of the pattern matching.
    \item The algorithm returns a debugging tree when it receives a debugging tree with children nodes as input.
    This is proved by noting that:
    \begin{itemize}
        \item \verb|are_all_idk head| is a true proposition, since all children nodes of a debugging tree satisfy \verb|are_all_idk| 
        \item \verb|get_debugging_tree_from_tree| returns a debugging tree when it receives a \verb|are_all_idk| tree, which is the case for \verb|verb| 
    \end{itemize}
    \item The weight of \verb|get_debugging_tree_from_tree head| is smaller than the weight of the input tree.
    \begin{itemize}
        \item the weight of \verb|get_debugging_tree_from_tree head| is equal to the weight of \verb|head|, since \verb|get_debugging_tree_from_tree| always returns a tree with the same weight as its input tree.  
        \item the weight of \verb|head| is smaller than the weight of its parent.
    \end{itemize}
    \item The weight of the input tree is a well-founded measure.
        
    This is proven by noting that the weight function returns a natural number, which is well-founded.
\end{enumerate}

\todo[color=green]{Tengo 24 lemmas auxiliares que me han ayudado a demostrar el algoritmo generico. Los menciono?}To discharge the proof obligation of the generic debugging algorithm, twenty four auxiliary lemmas were defined.

\todo{Esta sección es exageradamente corta, tiene que desarrollarse mucho más, dando ideas intuitivas. Si la sección son 2 párrafos parece que no tiene mucho valor}

\section{Benchmarks}
In this section we present and discuss different time benchmarks of the execution of DDC. We also list the performance bottlenecks found.

\subsection{Quicksort benchmarks}
To gather benchmarks easily, we provide a quicksort implementation that gets its input vector from the command line.
Then, we generate random lists of different lengths and pipe these lists to quicksort. All data was gathered with an Intel Core i5-4570 CPU @ 3.20GHz, with 4 cores and 16Gb of RAM computer.

In Figure \ref{fig:vector_length_vs_time} we plot the different execution times.

\begin{itemize}
    \item \verb|Execution| represents the native execution of the quicksort program.
    \item \verb|Recording| represents the recording of the quicksort program with \verb|rr|.
    \item \verb|Tree building (main memory)| represents the building of the debugging tree with DDC, storing the tree in main memory.
    \item \verb|Tree building (database)| represents the building of the debugging tree with DDC, storing the tree in a \verb|MySql| database.
    \item \verb|Trace debugging (GDB)| represents the time \verb|GDB| needs to, after setting the three breakpoints, navigate until the end.
    
    The three breakpoints are set in:
    \begin{itemize}
        \item quicksort,
        \item partition, and
        \item swap.
    \end{itemize}
    \item \verb|Trace debugging (rr)| represents the time \verb|rr| needs to, after setting the three breakpoints, navigate until the end.
\end{itemize}

To reduce the time needed to build the debugging tree, we have created a branch in DDC (called \verb|database|) in which we stored the debugging tree in a database, as proposed in \cite{DDJ}.
However, this did not improve the time significantly, as can be seen in Figure~\ref{fig:vector_length_vs_time}. Storing the debugging tree in a database is 43 percent  faster than storing it in main memory (50 seconds versus 87 seconds for a input vector length of 256).

\begin{figure}[htbp]
    \centering
    \begin{gnuplot}[terminal=pdf]
    set logscale x
    set xlabel "Input vector length (elements)"
    set logscale y
    set ylabel "Time (nanoseconds)"
    set xrange [1:200000]
    set yrange [1000000:200000000000]
    plot 'Datos/executing_quicksort.dat' using 1:2 title "Execution", 'Datos/recording_quicksort.dat' using 1:2 title "Recording", 'Datos/tree_building_quicksort.dat' using 1:2 title "Tree building (main memory)", 'Datos/tree_building_database.dat' using 1:2 title "Tree building (database)", 'Datos/gdb_quicksort.dat' using 1:2 title "Trace debugging (GDB)", 'Datos/rr_quicksort.dat' using 1:2 title "Trace debugging (rr)"
    \end{gnuplot}
    \caption{Quicksort execution vs record vs tree building}
    \label{fig:vector_length_vs_time}
\end{figure}

In Figure \ref{fig:node_vs_time_quicksort} we plot the debugging tree building time depending on the number of suspect functions set, for the same input list, which in this case is a list of length 256.
As we can see, the time required for building the tree increases with the number of suspect functions set.
The three data points correspond to the following:
\begin{itemize}
    \item 3 suspect functions: quicksort, partition, and swap.
    \item 2 suspect functions: quicksort and partition.
    \item 1 suspect function: quicksort.
\end{itemize}
\begin{figure}[htbp]
    \centering
    \begin{gnuplot}[terminal=pdf]
    set logscale x
    set xlabel "Number of nodes"
    set logscale y
    set ylabel "Time (nanoseconds)"
    set xrange [200:2000]
    set yrange [10000000000:100000000000]
    plot 'Datos/nodes_vs_time_3_suspect.dat' using 1:2 title "3 suspect functions", 'Datos/nodes_vs_time_2_suspect.dat' using 1:2 title "2 suspect functions", 'Datos/nodes_vs_time_1_suspect.dat' using 1:2 title "1 suspect function",
    \end{gnuplot}
    \caption{Nodes vs time}
    \label{fig:node_vs_time_quicksort}
\end{figure}

In Table \ref{table:DDC_profile} we can see the profile results of building a debugging tree with DDC. In this case, the program input is a quicksort execution with input vector length 256, and three suspect functions set (quicksort, partition, and swap). 
 
We have identified two bottlenecks:
\begin{itemize}
    \item Reverse stepping into a returning function (24.926 seconds).
    
    This is needed to gather the object state, output arguments and global variables, denoted \(O_o\), \(O_a\), and \(O_g\) respectively in Chapter \ref{cap:definitions}.
    
    \item Getting the string representation of a value (16.063 seconds).
    
    This is needed to store the value (argument, object, global variable, or return value) into the database.
\end{itemize}
This two bottlenecks constitute around 75 percent of total execution time.
 
\begin{table}
\caption{DDC (database) profiling for quicksort with input vector length 256}
\label{table:DDC_profile}
\begin{tabular}{rl}
cumtime & filename:lineno(function)\\
\hline
54.184 & \{built-in method builtins.exec\}\\
54.184 & <string>:1(<module>)\\
54.184 & \{built-in method \_gdb.execute\}\\
54.184 & \{declarative\_debugger\}.py:505(invoke)\\
38.693 & \{declarative\_debugger\}.py:282(invoke)\\
24.926 & \{declarative\_debugger\}.py:272(\{reverse\_stepi\})\\
16.078 & \{declarative\_debugger\}.py:847(\{print\_value\})\\
16.063 & \{method 'format\_string'  of 'gdb.Value'  objects\}\\
9.641 & \{declarative\_debugger\}.py:352(invoke)\\
8.281 & \{declarative\_debugger\}.py:374(<listcomp>)\\
7.804 & \{declarative\_debugger\}.py:299(<listcomp>)\\
\end{tabular}
\end{table}

\subsection{Z3 benchmarks}
To test DDC with a industrial grade program, we chose the SMT solver Z3 \cite{z3}.
The execution used for these benchmarks is the simplest execution possible.
We pass the SMT file shown in Figure \ref{fig:SMT_file} to Z3:
\begin{figure}[htbp]
    \centering
    \caption{SMT file passed to Z3 for benchmarking}
    \label{fig:SMT_file}
    \begin{verbatim}
(assert true)
(check-sat)
    \end{verbatim}
\end{figure}

In Figure \ref{fig:node_vs_time_z3} we see the relationship between number of nodes versus execution time. We get the same relationship as in with quicksort data.

\begin{figure}[htbp]
    \centering
    \begin{gnuplot}[terminal=pdf]
    set logscale x
    set xlabel "Number of nodes"
    set logscale y
    set ylabel "Time (nanoseconds)"
    set xrange [1:400]
    set yrange [1000000000:500000000000]
    plot 'Datos/z3_nodes_vs_time.dat' using 1:2 notitle
    \end{gnuplot}
    \caption{Nodes vs time (Z3)}
    \label{fig:node_vs_time_z3}
\end{figure}
In Figure \ref{fig:bp_vs_time_z3} we plot the number of breakpoints set versus execution time. We compare stopping at breakpoints without doing anything (data labeled \verb|Plain breakpoints|) with creating a node each time a breakpoint is reached (data labeled \verb|Tree building|).
\begin{figure}[htbp]
    \centering
    \begin{gnuplot}[terminal=pdf]
    set logscale x
    set xlabel "Number of breakpoints"
    set logscale y
    set ylabel "Time (nanoseconds)"
    set xrange [2:30]
    set yrange [100000000:1000000000000]
    plot 'Datos/z3_bp_vs_time_ddc.dat' using 1:2 title "Tree building", 'Datos/z3_bp_vs_time_rr.dat' using 1:2 title "Plain breakpoints"
    \end{gnuplot}
    \caption{Breakpoints vs time (Z3)}
    \label{fig:bp_vs_time_z3}
\end{figure}
