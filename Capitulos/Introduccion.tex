\chapter{Introduction}
\label{cap:introduccion}

\chapterquote{Frase célebre dicha por alguien inteligente}{Autor}


El estudiante elaborará una memoria descriptiva del trabajo realizado, con una \textbf{extensión mínima recomendada de 50 páginas} incluyendo al menos una introducción, objetivos y plan de trabajo, resultados con una discusión crítica y razonada de los mismos, conclusiones y bibliografía empleada en la elaboración de la memoria.

La memoria se puede redactar en castellano o en inglés, pero en el primer caso la introducción y las conclusiones de la memoria tienen que traducirse también al inglés y aparecerán como capítulos \textbf{al final de la memoria}. En ambos casos, el título de la memoria aparecerá en castellano y en inglés.

Además del cuerpo principal describiendo el trabajo realizado, la memoria contendrá los siguientes elementos, que no computarán para el cálculo de la extensión mínima del trabajo:

\begin{itemize}
\item un resumen en inglés de media página, incluyendo el título en inglés,
\item ese mismo resumen en castellano, incluyendo el título en castellano,
\item una lista de no más de 10 palabras clave en inglés,
\item esa misma lista en castellano,
\item un índice de contenidos, y
\item una bibliografía.
\end{itemize}

La portada de la memoria deberá contener la siguiente información:

\begin{itemize}
\item "Máster en NOMBRE DEL MÁSTER, Facultad de Informática, Universidad Complutense de Madrid"
\item Título
\item Autor
\item Director(es)
\item Colaborador externo de dirección, si lo hay
\item Curso académico
\item Solo en la versión final: convocatoria y calificación obtenida
\end{itemize}

Para facilitar la escritura de la memoria siguiendo esta estructura, el estudiante podrá usar las plantillas en LaTeX o Word preparadas al efecto y publicadas en la página web del máster correspondiente.

Todo el material no original, ya sea texto o figuras, deberá ser convenientemente citado y referenciado. En el caso de material complementario se deben respetar las licencias y copyrights asociados al software y hardware que se emplee. En caso contrario no se autorizará la defensa, sin menoscabo de otras acciones que correspondan.


\section{Motivation}
Introducción al tema del TFM.


\section{Objectives}
Descripción de los objetivos del trabajo.


\section{Work plan}
Aquí se describe el plan de trabajo a seguir para la consecución de los objetivos descritos en el apartado anterior.



\section{Explicaciones adicionales sobre el uso de esta plantilla}
Si quieres cambiar el \textbf{estilo del título} de los capítulos, edita \verb|TeXiS\TeXiS_pream.tex| y comenta la línea \verb|\usepackage[Lenny]{fncychap}| para dejar el estilo básico de \LaTeX.

Si no te gusta que no haya \textbf{espacios entre párrafos} y quieres dejar un pequeño espacio en blanco, no metas saltos de línea (\verb|\\|) al final de los párrafos. En su lugar, busca el comando  \verb|\setlength{\parskip}{0.2ex}| en \verb|TeXiS\TeXiS_pream.tex| y aumenta el valor de $0.2ex$ a, por ejemplo, $1ex$.

TFMTeXiS se ha elaborado a partir de la plantilla de TeXiS\footnote{\url{http://gaia.fdi.ucm.es/research/texis/}}, creada por Marco Antonio y Pedro Pablo Gómez Martín para escribir su tesis doctoral. Para explicaciones más extensas y detalladas sobre cómo usar esta plantilla, recomendamos la lectura del documento \texttt{TeXiS-Manual-1.0.pdf} que acompaña a esta plantilla.

El siguiente texto se genera con el comando \verb|\lipsum[2-20]| que viene a continuación en el fichero .tex. El único propósito es mostrar el aspecto de las páginas usando esta plantilla. Quita este comando y, si quieres, comenta o elimina el paquete \textit{lipsum} al final de \verb|TeXiS\TeXiS_pream.tex|

\subsection{Texto de prueba}


\lipsum[2-20]
