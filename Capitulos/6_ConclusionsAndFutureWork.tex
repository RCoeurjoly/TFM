\chapter{Conclusions and Future Work}
\label{cap:conclusions}

We now proceed to discuss the contributions made, and how they compare to our initial goals.

\section{Conclusions}
We have successfully developed a declarative debugger for C++, fulfilling all its initial goals.
In particular, we have achieved Goal \ref{goal1} (integrated in workflow) by using GDB and rr as the foundation to DDC. GDB is arguably the most used C++ debugger in GNU/Linux and rr enhances GDB\footnote{https://rr-project.org/}. Once familiar with GDB, the commands provided by DDC can be easily inspected and interacted with. For example, when issuing the \verb|suspect-function| command, auto-completion is enabled. Also, once the command is issued, the resulting breakpoint can be handled like any other breakpoint. This can be seen in figure \ref{fig:bpInteraction}, where after setting the suspect function (with command \verb|suspect-function swap(int*, int*)|), we check that it appears in the list of breakpoints (\verb|info breakpoints|), then disable it (\verb|disable 1|) and finally delete it (\verb|delete 1|).
\begin{figure}[h]
    \centering
    \caption{Setting, checking, disabling and deleting a suspect function}
    \label{fig:bpInteraction}
    \begin{verbatim}
(rr) suspect-function swap(int*, int*) 
Breakpoint 1 at 0x401222: file quicksort.h, line 9.
(rr) info breakpoints
Num Type           Disp Enb Address  What
1   breakpoint     keep y   0x401222 in swap(int*, int*) at quicksort.h:9
    add-node-to-session swap(int*, int*)
(rr) disable 1
(rr) info breakpoints
Num Type           Disp Enb Address  What
1   breakpoint     keep n   0x401222 in swap(int*, int*) at quicksort.h:9
    add-node-to-session swap(int*, int*)
(rr) delete 1
(rr) info breakpoints
No breakpoints or watchpoints.
    \end{verbatim}
\end{figure}

During the navigation phase, DDC provides the user with a intuitive command line interface, avoiding a context switch, which can be as disruptive as an interruption \cite{taskSwitching}.
More importantly, an algorithmic debugging session can be intertwined with a generic debugging session in any way, that is, the user can start the algorithmic debugging session after inspecting the program and vice versa. With this functionality, we address Issue-6 raised in the survey \cite{Survey} by \begin{quote}
    allowing the user to switch from a trace debugger to the algorithmic debugger easily
\end{quote}.

We also consider Goal \ref{goal2} (no changes to program) as achieved. No code instrumentation, such as \verb|# include| or \verb|# define| directives, is needed to debug a program with DDC. Only compiling with debug symbols is required.

Goal \ref{goal3} (tree transformations) has been fulfilled by implementing the simplified tree compression algorithm. These tree transformation reduces the tree size of recursive programs, therefore reducing the amount of questions the user has to answer. Also, we have build the necessary infrastructure to integrate other tree transformation painlessly in the future.

The final goal, Goal \ref{goal4} (test cases as oracles), has also been accomplished. Although another debugging session has to be started to send the correct nodes to the client, we consider that this is convenient enough not to deter its usage. The collection of the correct nodes directly, without starting another debugging session, would be possible if rr supported changing the executable in a running session (analog to the \verb|run| command in GDB), or if we removed rr as a dependency. However, we find it unlikely to remove the rr dependency since GDB support for reverse-stepping \cite{ReverseExecution_GDB}, needed to gather arguments when returning, is much less robust than in rr.

Lastly, a notable feature of DDC not included in the initial set of goals is the ability to support several languages \cite{SupportedLanguaged_GDB}, not only C++. This has been accomplished by the use of the GDB Python API, which uses abstractions like frames, symbols and variables that are common to all languages supported by GDB. Although more work is needed to fully support languages other than C++ (see Future work below), we estimate that the effort to do so is comparatively small.
\section{Future work}
During the research, development and use of DDC, we have found the following research directions to pursue in the future.

\subsection{Support building tree without suspecting a function or method}
So far, the user has to suspect at least one function or method before DDC can build the debugging tree.
Ideally, DDC would be able to build the DT without any suspect breakpoints, by creating a node every time it enters a new function.
Since we already have functionality to prune the DT of excess nodes (by trusting functions and using test cases as oracles, for example), the user could them reduce the size of the DT.

\subsection{Benchmarking overhead of building debugging tree}
As mentioned in \cite{Survey}, the speed and memory footprint of building the debugging tree is one of the most important metrics in the development of algorithmic debuggers.
We should collect a sample of real, large C++ programs and measure the time and memory needed to build their respective debugging trees.
\subsection{Formally verify all algorithms}
We made an effort to formally verify all algorithms used in DDC.
In Nagini \cite{nagini}, the tool we selected to perform the verification, we identified a bug\footnote{https://github.com/marcoeilers/nagini/issues/150}, which was confirmed by Nagini main author, that blocked further work. 
Another option worth considering would be to develop the algorithms in a programming language such as Lean \cite{lean}, which is also a theorem prover in which you can prove programs correct.
Lean can generate code in C, which could be call from Python.
\subsection{Support for C-style arrays}
C++ arrays are inherited from C to maintain backward compatibility.
As illustrated by figure \ref{fig:array_to_pointer}, when entering a function, an array changes its type to become a pointer to the type of the elements of the array.
In this example, \verb|arr| is of type \verb|int [6]| before entering \verb|quickSort| and becomes \verb|int *| on entry.
This makes it impossible to display an array to the user safely, since we do not know its size.
To avoid dealing with this issue, the examples programs provided with DDC use std::vector instead of C-style arrays.
\begin{figure}[h]
    \centering
    \caption{Integer array variable changes to integer pointer upon entry to function}
    \label{fig:array_to_pointer}
    \begin{verbatim}
(rr) ptype arr
type = int [6]
(rr) c
Continuing.
(rr) ptype arr
type = int *
    \end{verbatim}
\end{figure}
\subsection{Test programming languages other than C++}
The Rust code in listing \ref{lst:rustQuickSort} produces the node represented in figure \ref{fig:rustQuickSortTree}.
\begin{lstlisting}[language=Python, caption=QuickSort in Rust, frame=tb, label={lst:rustQuickSort}]
pub fn quick_sort<T: Ord>(arr: &mut [T]) {
    let len = arr.len();
    _quick_sort(arr, 0, (len - 1) as isize);
}
\end{lstlisting}
\begin{figure}[h]
\caption{Tree of quicksort using C-style arrays}
\label{fig:rustQuickSortTree}
\begin{verbatim}
quicksort::quick_sort<i32>                                                                                                                            
├── correctness                                                                                                                                       
│   └── I don't know                                                                                                                                  
├── weight                                                                                                                                            
│   └── 3                                                                                                                                             
└── args on entry                                                                                                                                     
    └── arr = &mut  {                                                                                                                                 
          data_ptr: 0x7ffd36788d80,                                                                                                                   
          length: 10                                                                                                                                  
        }
\end{verbatim}
\end{figure}
This tree has two problems:
\begin{itemize}
    \item Although it has already returned and the argument arr is mutable, the tree has no \verb|args when returning| branch.
    \item The representation of arr is not very helpful to the user, especially \verb|data_ptr|.
\end{itemize}
The \verb|args when returning| branch is created when the following function returns a non empty list:
\begin{lstlisting}[language=Python, caption=Python function to select arguments passed as reference or pointer, frame=tb]
def get_pointer_or_ref(arguments, frame):
    if arguments is None:
        return []
    return [argument for argument in arguments
            if argument.value(frame).type.code in [gdb.TYPE_CODE_PTR,
                                                   gdb.TYPE_CODE_REF]]
\end{lstlisting}
Therefore, we must deduce that the argument \verb|arr| is neither a pointer nor a reference. We should find out which type it is to support this Rust program.
\subsection{Support concurrent programs}
Currently, DDC does not support the debugging of concurrent programs.
This is mainly due to the fact that rr does not support multi threaded programs \cite{rr}.
We could remove the rr dependency to support concurrency, but reverse execution in GDB \cite{ReverseExecution_GDB} is more limited than in rr.
\subsection{Implement more strategies}
As mentioned in the survey \cite{Survey}, several navigation strategies have been developed.
Of special interest is Optimal Divide and Query \cite{optimal_strategy}, which has been proven correct and optimal.
One difficulty we may face when developing this strategy is that another weight marking has to be created for the debugging tree.
\subsection{Generate test cases from correct nodes}
When the user confirms the correctness of a certain node, a test case should be generated.
This presents us with several challenges:
\begin{itemize}
    \item Once the test case is create, should we present them via terminal to the user or insert them in a file? If inserting them is desired, should we create a new test file or insert them in an existing one?
    \item Creating the objects and variables requires a deep understanding of C++. Using the Clang Python API from LLVM \cite{llvm} may be needed.
    \item Testing private methods require creating a friend class to access them. Another option would be to skip them altogether.
\end{itemize}


%\subsection{Metamorphic Testing of GDB}
%\cite{MT_debuggers}
%\subsection{Increase granularity in error detection}
%\subsection{Increase flexibility inside a debugging session}
%\subsection{Interactive/collapsible debugging tree}