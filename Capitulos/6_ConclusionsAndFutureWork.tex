\chapter{Conclusions and Future Work}
\label{cap:conclusions}

Conclusions.

By using GDB as a framework to build the declarative debugger, we have the benefit of supporting not only C++ but several programming languages (https://sourceware.org/gdb/current/onlinedocs/gdb/Supported-Languages.html).

This is done through a common Python API.

Also, by using a language with a broad amount of libraries such as Python, the user interface and execution tree representation have been easy to implement.

During the development and use of the debugger, some execution trees did not match expectations.

Upon close examination, these maybe caused by GDB.

The first is that with recursive calls

The second is that if two function calls are identical in terms of arguments passed to them, then their frames are identical.

The workaround implemented to avoid building an erroneous execution tree is two fold:

When filling information about the call on entry, the current node is appended to the list of node if:
nodes is empty
last node's frame is invalid (meaning is no longer active)
last node's frame is not a parent of current node.

When filling information about the call on return, only fill it if it is empty.

Another future line of improvements could come from the presentation of node information to the user.

Two issues have been highlighted by doing this project.

The first is arrays in C++. C++ arrays are typed as pointers to the first element of the array when passed as argument to a function. (insert screen capture of GDB to prove this)

Then, it does not seem to be a way to differentiate between pointers and arrays when working in GDB.

To surmount this difficulty, the examples programs use std::vector instead of C-style arrays.

Another issue with variable display is pointers.

Some functions, like swap(int *, int *) in the example provided, do pointer arithmetic.

Maybe we should display both the memory address pointed by the pointer and the content.

Another line of future development could be 

\section{Conclusions}
\section{Future work}
\subsection{Benchmarking overhead of building execution tree}
\subsection{Supporting concurrency}
\subsection{Support for more strategies}
\subsection{Support for C-style arrays}
\subsection{Report GDB bugs}
\subsubsection{Wrong backtrace with recursive functions}
\subsubsection{Frame ID is the same for function call with the same arguments}
\subsection{Increase granularity in error detection}
\subsection{Increase flexibility inside a debugging session}
\subsection{Interactive/collapsible execution tree}